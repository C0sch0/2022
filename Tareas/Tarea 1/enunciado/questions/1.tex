%!TEX root = ../main/main.tex

Dado un alfabeto $\Sigma=\{\sigma_0,\ldots,\sigma_{N-1}\}$ definiremos el esquema criptográfico $\text{RP}^{\Sigma, \ell}$ (Repeated Pad sobre $\Sigma$ con llaves de largo $\ell$), como $(Gen, Enc, Dec)$ donde:
\begin{itemize}
  \item $Gen$ es la distribución uniforme sobre $\Sigma^\ell$.
  \item Dado $m=\sigma_{m_0},\ldots,\sigma_{m_{n-1}}\in\Sigma^*$ y $k=\sigma_{k_0},\ldots,\sigma_{k_{\ell-1}}\in\Sigma^\ell$, el texto cifrado $c=Enc(k, m)$ se define como $c=\sigma_{c_0},\ldots,\sigma_{c_{n-1}}$ donde $c_i=(\,m_i + k_{(i \!\mod \ell)}\,)\mod N$.
  \item Dado $c=\sigma_{c_0},\ldots,\sigma_{c_{n-1}}\in\Sigma^*$ y $k=\sigma_{k_0},\ldots,\sigma_{k_{\ell-1}}\in\Sigma^\ell$, el texto plano $m=Dec(k, c)$ se define como $m=\sigma_{m_0},\ldots,\sigma_{m_{n-1}}$ donde $m_i=(\ c_i - k_{(i\!\mod \ell)}\,)\mod N$.
  
\end{itemize}

En clases se mostró una forma de decriptar mensajes suficientemente largos encriptados con $\text{RP}^{\Sigma,\ell}$, suponiendo que los mensajes originales estaban escritos en inglés y $\ell$ era un valor conocido. Para esto se utilizó la \href{https://en.wikipedia.org/wiki/Letter_frequency}{frecuencia de caracteres} del inglés, además de una noción que definía intuitivamente cuánto dista un string de seguir dicha frecuencia. En esta pregunta se deberá generalizar lo hecho en clases para intentar decriptar $\text{RP}^{\Sigma, \ell}$ suponiendo lo siguiente:

\begin{itemize}
  \item El largo de la llave es desconocido.
  \item El mensaje original está en un idioma arbitrario, para el cual la frecuencia es conocida.
  \item La noción de cuánto dista un string de seguir una frecuencia de caracteres es arbitraria.
\end{itemize}

En concreto, deberá escribir una función que reciba (1) un texto cifrado, (2) una frecuencia de caracteres, y (3) una función que indica cuánto dista un string de seguir una frecuencia de caracteres, y retorne la llave que se utilizó para encriptar el texto cifrado.

La entrega de esta pregunta deberá seguir las instrucciones indicadas más arriba, entregando un Jupyter Notebook que defina una función como la siguiente:

\bigskip
\begin{python}
  def break_rp(
      ciphertext: str,
      frequencies: {str: float},
      distance: (str, {str: float}) -> float,
    ) -> str:
    """
    Arguments:
      ciphertext: An abritrary string representing the
                  encrypted version of a plaintext.
      frequencies: A dictionary representing a character
                   frequency over the alphabet.
      distance: A function indicating how distant is a string
                from following a character frequency
    Returns:
      key: A guess of the key used to encrypt the ciphertext, assuming
           that the plaintext message was written in a language in which
           letters distribute according to frequencies.
    """
\end{python}

Para probar su respuesta se recomienda utilizar la distancia que definimos en clases, dada por la siguiente función:

\bigskip
\begin{python}
  def abs_distance(string: str, frequencies: {str: float}) -> float:
    """
    Arguments:
      string: An abritrary string
      frequencies: A dictionary representing a character frequency
    Returns:
      distance: How distant is the string from the character frequency
    """
    return sum([
      abs(frequencies[c] - string.count(c) / len(string))
      for c in frequencies
    ])
\end{python}
Puede suponer que el alfabeto es el conjunto de llaves del diccionario \texttt{frequencies}, y que el largo de la llave es a lo más el largo de \texttt{ciphertext} dividido en 50.

\textbf{Restricción.} Si se utiliza la función \texttt{abs\_distance} definida arriba, un texto cifrado con 1000 caracteres y un alfabeto de 30 caracteres, su función no puede demorar más de 10 segundos en su propio computador.
