%!TEX root = ../main/main.tex

\section*{Instrucciones}

Cualquier duda sobre la tarea se deberá hacer en los \emph{issues} del \href{https://github.com/UC-IIC3253/2022}{repositorio del curso}. Si quiere usar alguna librería en sus soluciones debe preguntar primero si dicha librería está permitida. El foro es el canal de comunicación oficial para todas las tareas.

\paragraph{Configuración inicial.} Para entregar las tareas usted deberá crear un repositorio en \href{https://github.com}{GitHub} y darle acceso como colaboradores a los ayudantes del curso, cuyos usuarios son \texttt{@johntrombon} y \texttt{@jnhasard}. También deberá responder \href{https://docs.google.com/forms/d/e/1FAIpQLScf4rDsePbU20UyzUxsD6lOi1SU_gOc5c6kQv39E2cPCrmwpQ/viewform?usp=sf_link}{este formulario}, en el que se le pedirá una \textbf{llave simétrica} que será utilizada para encriptar sus notas y correcciones usando el esquema \href{https://en.wikipedia.org/wiki/Advanced_Encryption_Standard}{AES}. En el repositorio del curso se publicará además el valor de hash de su llave utilizando \href{https://en.wikipedia.org/wiki/SHA-2}{SHA256}. Si usted pierde dicha llave podrá recuperarla incurriendo en una penalización de dos décimas en el promedio de sus tareas. Si otra persona descubre su llave antes de que se publiquen las notas de la tercera tarea, el promedio de tareas de dicha persona aumentará en cinco décimas, mientras que el suyo disminuirá en cinco décimas. Recomendación: use un administrador de contraseñas.

\paragraph{Entrega.} Al entregar esta tarea, su repositorio se deberá ver exactamente de la siguiente forma:

\bigskip

\dirtree{%
.1 \faGithub \ Repositorio.
.2 \faFolderOpenO \ Tarea 1.
.3 \faFolderOpenO \ Pregunta 1.
.4 \faFileTextO \ \texttt{requirements.txt}.
.4 \faFileCodeO \ \texttt{pregunta1.ipynb}.
.3 \faFolderOpenO \ Pregunta 2.
.4 \faFileTextO \ \texttt{requirements.txt}.
.4 \faFileCodeO \ \texttt{pregunta2.ipynb}.
.3 \faFolderOpenO \ Pregunta 3.
.4 \faFilePdfO \ \texttt{pregunta3.pdf}.
.3 \faFolderOpenO \ Pregunta 4.
.4 \faFilePdfO \ \texttt{pregunta4.pdf}.
.2 \faFileTextO \ \texttt{.gitignore}.
.2 \faFileTextO \ \texttt{README.md}.
.2 \faFolderO \ .git.
}

\bigskip

Deberá considerar lo siguiente:

\begin{itemize}

  \item El archivo \texttt{requirements.txt} dentro de la carpeta de una pregunta deberá especificar todas las librerías que se necesitan instalar para ejecutar el código de su respuesta a dicha pregunta. Este archivo debe seguir la especificación de \href{https://pypi.org/project/pip/}{Pip}, es decir se debe poder ejecutar el comando \texttt{pip install -r requirements.txt} suponiendo una versión de \texttt{Pip} mayor o igual a 20.0 que apunta a la versión 3.9 de Python. Si su respuesta no requiere librerías adicionales, este archivo debe estar vacío (pero debe estar en su repositorio).

\item La solución de cada problema de programación debe ser entregada como un Jupyter Notebook (esto es, un archivo con extensión \texttt{ipynb}). Este archivo debe contener comentarios que expliquen claramente el razonamiento tras la solución del problema, idealmente utilizando \emph{markdown}. Más aun, su archivo deberá ser exportable a un módulo de python utilizando el comando de consola
\begin{verbatim}
    jupyter-nbconvert --to python preguntaX.ipynb
\end{verbatim}
Este comando generará un archivo \texttt{preguntaX.py}, del cual se deben poder importar las funciones y clases que se piden en cada pregunta. Por ejemplo, para la Pregunta 1, luego de ejecutar este comando, se debe poder importar desde otro archivo python (ubicado en el mismo directorio) la función \texttt{break\_rp} simplemente con \texttt{from pregunta1 import break\_rp}.

\item Para cada problema cuya solución se deba entregar como un documento (en este caso las preguntas 2 y 4), usted deberá entregar un archivo \texttt{.pdf} que, o bien fue construido utilizando \LaTeX, o bien es el resultado de digitalizar un documento escrito a mano. En caso de optar por esta última opción, queda bajo su responsabilidad la legibilidad del documento. Respuestas que no puedan interpretar de forma razonable los ayudantes y profesores, ya sea por la caligrafía o la calidad de la digitalización, serán evaluadas con la nota mínima.

\end{itemize}
